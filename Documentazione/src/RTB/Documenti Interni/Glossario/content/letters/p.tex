\begin{center}\section*{P}\end{center}
\addcontentsline{toc}{section}{P}

\term{Patch}
Una modifica atta a risolvere ortografici o di struttura. In un sistema di versionamento x.y.z, la patch è rappresentata dalla lettera z, e ogni modifica di tipo patch
aggiorna la versione di un file da x.y.z a x.y.(z+1).
\term{PB}
Acronimo di \textit{Product Baseline}, per la definizione vedere 'Product Baseline'
\term{PoC}
Acronimo di \textit{Proof of Concept}, per la definizione vedere 'Proof of Concept'
\term{PR}
Acronimo di \textit{Pull Request}, per la definizione vedere 'Pull Request'
\term{Product Baseline}
Fase finale in un progetto, che prevede la realizzazione di un Minimum Viable Product, a seguito del superamento dei test concordati nella 
Requirements and Technology Baseline.
\term{Proof of Concept}
Prodotto dimostrativo atto a presentare la fattibilità dell'idea.
\term{Pull Request}
All'interno dell'ecosistema GitHub, uno strumento che permette di sospendere momentaneamente la pubblicazione di una modifica nel ramo principale della repository, in attesa 
di approvazione e/o workflow automatici (e.g. GitHub Action).